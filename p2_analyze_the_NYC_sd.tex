\documentclass{article}
%% Packages:
    \usepackage{graphicx} % Used to insert images
    \usepackage{adjustbox} % Used to constrain images to a maximum size 
    \usepackage{xcolor} % Allow colors to be defined
    \usepackage{setspace}
    \usepackage{minted}
    \newminted[mysql1]{mysql}{
      mathescape,
      linenos=true, 
      texcl=true, 
      bgcolor=Background
      %gobble=2,
    }

    %\usepackage{listings}
    \usepackage{float}
    \usepackage{enumerate} % Needed for markdown enumerations to work
    \usepackage{geometry} % Used to adjust the document margins
    \usepackage{amsmath} % Equations and \numberwithin command
    \usepackage{amssymb} % Equations
    \usepackage[T1]{fontenc}
    \usepackage{mdframed}
    \usepackage{newtxtext,newtxmath}
    \usepackage[utf8]{inputenc} % Allow utf-8 characters in the tex document
    \usepackage{fancyvrb} % verbatim replacement that allows latex
    \usepackage{grffile} % extends the file name processing of package graphics 
                         % to support a larger range 
    % The hyperref package gives us a pdf with properly built
    % internal navigation ('pdf bookmarks' for the table of contents,
    % internal cross-reference links, web links for URLs, etc.)
    \usepackage{tikz}
    \usepackage{pgfplots}
    \usepackage[]{ulem}
    \pgfplotsset{compat=newest}
    \usepackage{url}
    \usepackage{longtable} % longtable support required by pandoc >1.10
    \usepackage{booktabs}  % table support for pandoc > 1.12.2
    \usepackage{ulem} % ulem is needed to support strikethroughs (\sout)
    \usepackage{titlesec}
    %\titleformat{\section}{\normalfont\Large\bfseries}{Question {\thesection}: }{1em}{}
    \usepackage{hyperref}

    \renewcommand{\sectionautorefname}{\S}
    % Define a nice break command that doesn't care if a line doesn't already
    % exist.
    \def\br{\hspace*{\fill} \\* }
    % Math Jax compatability definitions
    \def\gt{>}
    \def\lt{<}
    % Document parameters

    \title{Analyzing the NYC Subway Dataset}
    \author{Yigal Weinstein}

    \sloppy 
    \usepackage[
      autocite=inline, 
      backend=biber,
      labeldate=true, 
      refsegment=section,
      uniquename=full,
      defernumbers=true,
      uniquelist=true]
    {biblatex}
    \addbibresource{udacity.bib}
    \PassOptionsToPackage{unicode}{hyperref}
    \PassOptionsToPackage{naturalnames}{hyperref}
    \hypersetup{pdfauthor={Yigal Weinstein},
      pdftitle={P2: Analyzing the NYC Subway Dataset},
      pdfsubject={data science, Udacity, MOOC},
      pdfkeywords={data science, statistics, probability, machine learning},
      pdfcreator={PdfLaTeX},
      bookmarks={true},            %  A set of Acrobat bookmarks are written
      colorlinks={true},           %  Colors the text of links and anchors. 
      anchorcolor={black},         %  Color for anchor text.
      filecolor={cyan},            %  Color for URLs which open local files.
      menucolor={red},             %  Color for Acrobat menu items.
      runcolor={blue},              %  Color for run links (launch annotations).        
      linkcolor={red!50!black},
      citecolor={blue!50!black},
      urlcolor={blue!80!black},
      breaklinks=true,  % so long urls are correctly broken across lines
    }           

    \definecolor{Background}{rgb}{0.95,0.95,0.92}
    %\geometry{verbose,tmargin=1in,bmargin=1in,lmargin=1in,rmargin=1in}

\makeatletter

\def\@seccntformat#1{%
  \expandafter\ifx\csname c@#1\endcsname\c@section\else
  \csname the#1\endcsname\quad
  \fi}
  \newcommand{\ssection}[1]{
\section*{#1}
\addcontentsline{toc}{section}{\protect\numberline{}#1}
}
\newcounter{questionCtr}

\newenvironment{question}{%      define a custom environment
   \bigskip\noindent%         create a vertical offset to previous material
   \refstepcounter{questionCtr}% increment the environment's counter
   \textsc{Question \thequestionCtr}% or \textbf, \textit, ...
   \newline%
   }{\par\bigskip}  %          create a vertical offset to following material
\numberwithin{questionCtr}{section}

\newcounter{problemCtr}

\newenvironment{problem}{%      define a custom environment
   \bigskip\noindent%         create a vertical offset to previous material
   \refstepcounter{problemCtr}% increment the environment's counter
   \textsc{Problem \theproblemCtr}% or \textbf, \textit, ...
   \newline%
   }{\par\bigskip}  %          create a vertical offset to following material
\numberwithin{problemCtr}{section}

\makeatother

\begin{document}
\maketitle
\section*{Introduction}
This project attempts to follow the outline provided in \cite{Udacity-P2-DS}
and the project rubrick set forth in \cite{Udacity-P2-rubrick}.  It is divided
into two parts.  The first part, below, are answers to the short questions in
\cite{Udacity-P2-DS} while the second part contains the code and and notes
regarding problem set 2,3, and 4 of the Intro to Data Science course
\cite{Udacity-data-science-course}.
\part{Analyzing the NYC Subway Dataset}
\section{Statistical Test}
\label{sec:statistical_test_one}
\begin{question}
  Which statistical test did you use to analyze the NYC subway data? Did you use
  a one-tail or a two-tail P value? What is the null hypothesis? What is your
  p-critical value?
\end{question}

\begin{question}
  Why is this statistical test applicable to the dataset? In particular,
  consider the assumptions that the test is making about the distribution of
  ridership in the two samples.
\end{question}

\begin{question}
  What results did you get from this statistical test? These should include the
  following numerical values: p-values, as well as the means for each of the two
  samples under test.
\end{question}

\begin{question}
  What is the significance and interpretation of these results?
\end{question}

\section{Linear Regression}

\begin{question}
  What approach did you use to compute the coefficients theta and produce
  prediction for \verb|ENTRIESn_hourly| in your regression model:
  \begin{itemize}
    \item OLS using Statsmodels or Scikit Learn
    \item Gradient descent using Scikit Learn
    \item Or something different?
  \end{itemize}
\end{question}

\begin{question}
  What features (input variables) did you use in your model? Did you use any
  dummy variables as part of your features?
\end{question}

\begin{question}
  Why did you select these features in your model? We are looking for specific
  reasons that lead you to believe that the selected features will contribute to
  the predictive power of your model.
  \begin{itemize}
    \item Your reasons might be based on intuition. For example, response for
      fog might be: “I decided to use fog because I thought that when it is very
      foggy outside people might decide to use the subway more often.”
    \item Your reasons might also be based on data exploration and
      experimentation, for example: “I used feature X because as soon as I
      included it in my model, it drastically improved my R2 value.”
  \end{itemize}
\end{question}

\begin{question}
  What are the parameters (also known as ``coefficients'' or ``weights'') of the
  non-dummy features in your linear regression model?
\end{question}

\begin{question}
  What is your model’s $R^2$ (coefficients of determination) value?
\end{question}

\begin{question}
  What does this $R^2$ value mean for the goodness of fit for your regression
  model? Do you think this linear model to predict ridership is appropriate for
  this dataset, given this $R^2$  value?
\end{question}

\section{Visualization}
Please include two visualizations that show the relationships between two or
more variables in the NYC subway data.
Remember to add appropriate titles and axes labels to your plots. Also, please
add a short description below each figure commenting on the key insights
depicted in the figure.

One visualization should contain two histograms: one of  \verb|ENTRIESn_hourly| for
rainy days and one of \verb|ENTRIESn_hourly| for non-rainy days.
\begin{question}
  One visualization should contain two histograms: one of  \verb|ENTRIESn_hourly| for
  rainy days and one of \verb|ENTRIESn_hourly| for non-rainy days.
\begin{itemize}
  \item One visualization should contain two histograms: one of  \verb|ENTRIESn_hourly|
    for rainy days and one of \verb|ENTRIESn_hourly| for non-rainy days.
  \item If you decide to use to two separate plots for the two histograms,
    please ensure that the x-axis limits for both of the plots are identical. It
    is much easier to compare the two in that case.
  \item For the histograms, you should have intervals representing the volume of
    ridership (value of \verb|ENTRIESn_hourly|) on the x-axis and the frequency of
    occurrence on the y-axis. For example, each interval (along the x-axis), the
    height of the bar for this interval will represent the number of records
    (rows in our data) that have \verb|ENTRIESn_hourly| that falls in this interval.
  \item Remember to increase the number of bins in the histogram (by having
    larger number of bars). The default bin width is not sufficient to capture
    the variability in the two samples.
\end{itemize}
\end{question}

\begin{question}
  One visualization can be more freeform. You should feel free to implement
  something that we discussed in class (e.g., scatter plots, line plots) or
  attempt to implement something more advanced if you'd like. Some suggestions
  are:
  \begin{itemize}
    \item Ridership by time-of-day
    \item Ridership by day-of-week
  \end{itemize}
\end{question}

\section{Conclusion}
Please address the following questions in detail. Your answers should be 1-2
paragraphs long.
\begin{question}
 From your analysis and interpretation of the data, do more people ride
 the NYC subway when it is raining or when it is not raining?
\end{question}

\begin{question}
  What analyses lead you to this conclusion? You should use results from both
  your statistical tests and your linear regression to support your analysis.
\end{question}

\section{Reflection}
Please address the following questions in detail. Your answers should be 1-2
paragraphs long.

\begin{question}
Please discuss potential shortcomings of the methods of your analysis,
including:  
\begin{itemize}
  \item Dataset
  \item Analysis, such as the linear regression model or statistical test.
\end{itemize}
\end{question}

\begin{question}
  (Optional) Do you have any other insight about the dataset that you would like
  to share with us?
\end{question}
%\printbibliography[keyword=statistics , title={Statistics references}]
%\printbibliography[keyword=Matplotlib , title={Matplotlib references}]
%\printbibliography[keyword=SciPy , title={SciPy references}]
%\printbibliography[keyword=PANDAS , title={PANDAS references}]
\part{Code and Notes to Problem Sets 2,3 and 4 of \cite{Udacity-data-science-course}}
\ssection{Problem Set 2: Wrangling Subway Data}

% apparently the problems appear to have come from 
% http://usa-da.blogspot.com/2014/06/data-wrangling-project-2-wrangle-nyc.html
\begin{problem}
Number of Rainy Days
\end{problem}
The SQL query I used was,

\begin{mysql1}
SELECT COUNT(rain) FROM weather_data WHERE rain > 0.0
\end{mysql1}

\begin{problem}
  Temp on Foggy and Nonfoggy Days1 - Number of Rainy
\end{problem}

With the help of \cite{SQL-max} and \cite{Stackoverflow-sql-max} the following
query provides the desired result:

\begin{mysql1}
SELECT fog, MAX(maxtempi) FROM weather_data GROUP BY fog
\end{mysql1}

\begin{problem}
  Mean Temp on Weekends1 - Number of Rainy Days
\end{problem}
This one was extremely difficult as it required piecing together the hint about
using the \verb|CAST()| function in references to data in more than one column
as well as using the proper syntax to convert the \verb|date| column to the day
of the week to allow filtering on Saturday and Sunday.  Another challenge was
realizing that \verb|pandasql| isn't using \verb|SQL| per se but
\hyperref{http://www.sqlite.org/lang_datefunc.html}{SQLite}, and in this
particular case it matters because \verb|STRFTIME()| isn't something one will
use in \verb|MySQL|, but for this purpose it appears like the only function to
call. \cite{Python-strftime}  So continuing the research I found that SQLite
stores stores in ISO format as a string. \cite{SQL-sqlite-datatypes},
\cite{SQL-sqlite-date-time-functions}

\begin{mysql1}
SELECT AVG(CAST(meantempi AS INTEGER)) FROM weather_data WHERE
    CAST(STRFTIME('%w',date) as integer) = 6 OR
    CAST(STRFTIME('%w',date) as integer) = 0;
\end{mysql1}

\begin{problem}
  Mean Temp on Rainy Days1 - Number of Rainy Days
\end{problem}

\begin{mysql1}
SELECT AVG(CAST(mintempi AS INTEGER)) FROM weather_data WHERE
    CAST(rain AS INTEGER) == 1 and
    CAST(mintempi AS INTEGER) > 55; 
\end{mysql1}
No where near as challenging as the previous problem.

\begin{problem}
  
\end{problem}
\ssection{Problem Set 3: Analyzing Subway Data}
\ssection{Problem Set 4: Visualizing Subway Data}
\printbibliography[keyword=Udacity , title={Udacity references}]
\printbibliography[keyword=SQL , title={SQL references}]
%\printbibliography[keyword=LaTeX , title={{pdf\LaTeX}  references}]
%\printbibliography[keyword=other , title={other references}]
\end{document}
