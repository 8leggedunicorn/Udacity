    \usepackage{graphicx} % Used to insert images
    \usepackage{adjustbox} % Used to constrain images to a maximum size 
    \usepackage{xcolor} % Allow colors to be defined
    \usepackage{listings}
    \usepackage{setspace}
    \usepackage{float}
    \usepackage{enumerate} % Needed for markdown enumerations to work
    \usepackage{geometry} % Used to adjust the document margins
    \usepackage{amsmath} % Equations and \numberwithin command
    \usepackage{amssymb} % Equations
    \usepackage[T1]{fontenc}
    \usepackage{newtxtext,newtxmath}
    \usepackage[utf8]{inputenc} % Allow utf-8 characters in the tex document
    \usepackage{fancyvrb} % verbatim replacement that allows latex
    \usepackage{grffile} % extends the file name processing of package graphics 
                         % to support a larger range 
    % The hyperref package gives us a pdf with properly built
    % internal navigation ('pdf bookmarks' for the table of contents,
    % internal cross-reference links, web links for URLs, etc.)
    \usepackage{tikz}
    \usepackage{pgfplots}
    \usepackage[]{ulem}
    \pgfplotsset{compat=newest}
    \usepackage{url}
    \usepackage{longtable} % longtable support required by pandoc >1.10
    \usepackage{booktabs}  % table support for pandoc > 1.12.2
    \usepackage{ulem} % ulem is needed to support strikethroughs (\sout)
    \usepackage{titlesec}
    %\titleformat{\section}{\normalfont\Large\bfseries}{Question {\thesection}: }{1em}{}
    \usepackage[
      autocite=inline, 
      backend=biber,
      labeldate=true, 
      refsegment=section,
      uniquename=full,
      defernumbers=true,
      uniquelist=true]
    {biblatex}
    \addbibresource{udacity.bib}
    \usepackage{hyperref}
