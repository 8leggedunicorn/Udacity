\documentclass{article}
    \usepackage{graphicx} % Used to insert images
    \usepackage{adjustbox} % Used to constrain images to a maximum size 
    \usepackage{color} % Allow colors to be defined
    \usepackage{enumerate} % Needed for markdown enumerations to work
    \usepackage{geometry} % Used to adjust the document margins
    \usepackage{amsmath} % Equations
    \usepackage{amssymb} % Equations
    \usepackage{eurosym} % defines \euro
    \usepackage[mathletters]{ucs} % Extended unicode (utf-8) support
    \usepackage[utf8x]{inputenc} % Allow utf-8 characters in the tex document
    \usepackage{fancyvrb} % verbatim replacement that allows latex
    \usepackage{grffile} % extends the file name processing of package graphics 
                         % to support a larger range 
    % The hyperref package gives us a pdf with properly built
    % internal navigation ('pdf bookmarks' for the table of contents,
    % internal cross-reference links, web links for URLs, etc.)
    \usepackage{tikz}
    \usepackage{hyperref}
    \usepackage{url}
    \usepackage{longtable} % longtable support required by pandoc >1.10
    \usepackage{booktabs}  % table support for pandoc > 1.12.2
    \usepackage{ulem} % ulem is needed to support strikethroughs (\sout)
    \definecolor{orange}{cmyk}{0,0.4,0.8,0.2}
    \definecolor{darkorange}{rgb}{.71,0.21,0.01}
    \definecolor{darkgreen}{rgb}{.12,.54,.11}
    \definecolor{myteal}{rgb}{.26, .44, .56}
    \definecolor{gray}{gray}{0.45}
    \definecolor{lightgray}{gray}{.95}
    \definecolor{mediumgray}{gray}{.8}
    \definecolor{inputbackground}{rgb}{.95, .95, .85}
    \definecolor{outputbackground}{rgb}{.95, .95, .95}
    \definecolor{traceback}{rgb}{1, .95, .95}
    % ansi colors
    \definecolor{red}{rgb}{.6,0,0}
    \definecolor{green}{rgb}{0,.65,0}
    \definecolor{brown}{rgb}{0.6,0.6,0}
    \definecolor{blue}{rgb}{0,.145,.698}
    \definecolor{purple}{rgb}{.698,.145,.698}
    \definecolor{cyan}{rgb}{0,.698,.698}
    \definecolor{lightgray}{gray}{0.5}
    
    % bright ansi colors
    \definecolor{darkgray}{gray}{0.25}
    \definecolor{lightred}{rgb}{1.0,0.39,0.28}
    \definecolor{lightgreen}{rgb}{0.48,0.99,0.0}
    \definecolor{lightblue}{rgb}{0.53,0.81,0.92}
    \definecolor{lightpurple}{rgb}{0.87,0.63,0.87}
    \definecolor{lightcyan}{rgb}{0.5,1.0,0.83}
    
    % commands and environments needed by pandoc snippets
    % extracted from the output of `pandoc -s`
    \providecommand{\tightlist}{%
      \setlength{\itemsep}{0pt}\setlength{\parskip}{0pt}}
    \DefineVerbatimEnvironment{Highlighting}{Verbatim}{commandchars=\\\{\}}
    % Add ',fontsize=\small' for more characters per line
    \newenvironment{Shaded}{}{}
    \newcommand{\KeywordTok}[1]{\textcolor[rgb]{0.00,0.44,0.13}{\textbf{{#1}}}}
    \newcommand{\DataTypeTok}[1]{\textcolor[rgb]{0.56,0.13,0.00}{{#1}}}
    \newcommand{\DecValTok}[1]{\textcolor[rgb]{0.25,0.63,0.44}{{#1}}}
    \newcommand{\BaseNTok}[1]{\textcolor[rgb]{0.25,0.63,0.44}{{#1}}}
    \newcommand{\FloatTok}[1]{\textcolor[rgb]{0.25,0.63,0.44}{{#1}}}
    \newcommand{\CharTok}[1]{\textcolor[rgb]{0.25,0.44,0.63}{{#1}}}
    \newcommand{\StringTok}[1]{\textcolor[rgb]{0.25,0.44,0.63}{{#1}}}
    \newcommand{\CommentTok}[1]{\textcolor[rgb]{0.38,0.63,0.69}{\textit{{#1}}}}
    \newcommand{\OtherTok}[1]{\textcolor[rgb]{0.00,0.44,0.13}{{#1}}}
    \newcommand{\AlertTok}[1]{\textcolor[rgb]{1.00,0.00,0.00}{\textbf{{#1}}}}
    \newcommand{\FunctionTok}[1]{\textcolor[rgb]{0.02,0.16,0.49}{{#1}}}
    \newcommand{\RegionMarkerTok}[1]{{#1}}
    \newcommand{\ErrorTok}[1]{\textcolor[rgb]{1.00,0.00,0.00}{\textbf{{#1}}}}
    \newcommand{\NormalTok}[1]{{#1}}
    
    % Additional commands for more recent versions of Pandoc
    \newcommand{\ConstantTok}[1]{\textcolor[rgb]{0.53,0.00,0.00}{{#1}}}
    \newcommand{\SpecialCharTok}[1]{\textcolor[rgb]{0.25,0.44,0.63}{{#1}}}
    \newcommand{\VerbatimStringTok}[1]{\textcolor[rgb]{0.25,0.44,0.63}{{#1}}}
    \newcommand{\SpecialStringTok}[1]{\textcolor[rgb]{0.73,0.40,0.53}{{#1}}}
    \newcommand{\ImportTok}[1]{{#1}}
    \newcommand{\DocumentationTok}[1]{\textcolor[rgb]{0.73,0.13,0.13}{\textit{{#1}}}}
    \newcommand{\AnnotationTok}[1]{\textcolor[rgb]{0.38,0.63,0.69}{\textbf{\textit{{#1}}}}}
    \newcommand{\CommentVarTok}[1]{\textcolor[rgb]{0.38,0.63,0.69}{\textbf{\textit{{#1}}}}}
    \newcommand{\VariableTok}[1]{\textcolor[rgb]{0.10,0.09,0.49}{{#1}}}
    \newcommand{\ControlFlowTok}[1]{\textcolor[rgb]{0.00,0.44,0.13}{\textbf{{#1}}}}
    \newcommand{\OperatorTok}[1]{\textcolor[rgb]{0.40,0.40,0.40}{{#1}}}
    \newcommand{\BuiltInTok}[1]{{#1}}
    \newcommand{\ExtensionTok}[1]{{#1}}
    \newcommand{\PreprocessorTok}[1]{\textcolor[rgb]{0.74,0.48,0.00}{{#1}}}
    \newcommand{\AttributeTok}[1]{\textcolor[rgb]{0.49,0.56,0.16}{{#1}}}
    \newcommand{\InformationTok}[1]{\textcolor[rgb]{0.38,0.63,0.69}{\textbf{\textit{{#1}}}}}
    \newcommand{\WarningTok}[1]{\textcolor[rgb]{0.38,0.63,0.69}{\textbf{\textit{{#1}}}}}
    
    
    % Define a nice break command that doesn't care if a line doesn't already
    % exist.
    \def\br{\hspace*{\fill} \\* }
    % Math Jax compatability definitions
    \def\gt{>}
    \def\lt{<}
    % Document parameters
    
\title{Exploring the Stroop Effect utilizing IPython}
\author{Yigal Weinstein}

    % Pygments definitions
    
\makeatletter
\def\PY@reset{\let\PY@it=\relax \let\PY@bf=\relax%
    \let\PY@ul=\relax \let\PY@tc=\relax%
    \let\PY@bc=\relax \let\PY@ff=\relax}
\def\PY@tok#1{\csname PY@tok@#1\endcsname}
\def\PY@toks#1+{\ifx\relax#1\empty\else%
    \PY@tok{#1}\expandafter\PY@toks\fi}
\def\PY@do#1{\PY@bc{\PY@tc{\PY@ul{%
    \PY@it{\PY@bf{\PY@ff{#1}}}}}}}
\def\PY#1#2{\PY@reset\PY@toks#1+\relax+\PY@do{#2}}

\expandafter\def\csname PY@tok@ni\endcsname{\let\PY@bf=\textbf\def\PY@tc##1{\textcolor[rgb]{0.60,0.60,0.60}{##1}}}
\expandafter\def\csname PY@tok@gi\endcsname{\def\PY@tc##1{\textcolor[rgb]{0.00,0.63,0.00}{##1}}}
\expandafter\def\csname PY@tok@nd\endcsname{\def\PY@tc##1{\textcolor[rgb]{0.67,0.13,1.00}{##1}}}
\expandafter\def\csname PY@tok@nv\endcsname{\def\PY@tc##1{\textcolor[rgb]{0.10,0.09,0.49}{##1}}}
\expandafter\def\csname PY@tok@s2\endcsname{\def\PY@tc##1{\textcolor[rgb]{0.73,0.13,0.13}{##1}}}
\expandafter\def\csname PY@tok@ne\endcsname{\let\PY@bf=\textbf\def\PY@tc##1{\textcolor[rgb]{0.82,0.25,0.23}{##1}}}
\expandafter\def\csname PY@tok@ss\endcsname{\def\PY@tc##1{\textcolor[rgb]{0.10,0.09,0.49}{##1}}}
\expandafter\def\csname PY@tok@vi\endcsname{\def\PY@tc##1{\textcolor[rgb]{0.10,0.09,0.49}{##1}}}
\expandafter\def\csname PY@tok@c1\endcsname{\let\PY@it=\textit\def\PY@tc##1{\textcolor[rgb]{0.25,0.50,0.50}{##1}}}
\expandafter\def\csname PY@tok@vc\endcsname{\def\PY@tc##1{\textcolor[rgb]{0.10,0.09,0.49}{##1}}}
\expandafter\def\csname PY@tok@nf\endcsname{\def\PY@tc##1{\textcolor[rgb]{0.00,0.00,1.00}{##1}}}
\expandafter\def\csname PY@tok@kr\endcsname{\let\PY@bf=\textbf\def\PY@tc##1{\textcolor[rgb]{0.00,0.50,0.00}{##1}}}
\expandafter\def\csname PY@tok@nc\endcsname{\let\PY@bf=\textbf\def\PY@tc##1{\textcolor[rgb]{0.00,0.00,1.00}{##1}}}
\expandafter\def\csname PY@tok@vg\endcsname{\def\PY@tc##1{\textcolor[rgb]{0.10,0.09,0.49}{##1}}}
\expandafter\def\csname PY@tok@no\endcsname{\def\PY@tc##1{\textcolor[rgb]{0.53,0.00,0.00}{##1}}}
\expandafter\def\csname PY@tok@s1\endcsname{\def\PY@tc##1{\textcolor[rgb]{0.73,0.13,0.13}{##1}}}
\expandafter\def\csname PY@tok@ow\endcsname{\let\PY@bf=\textbf\def\PY@tc##1{\textcolor[rgb]{0.67,0.13,1.00}{##1}}}
\expandafter\def\csname PY@tok@na\endcsname{\def\PY@tc##1{\textcolor[rgb]{0.49,0.56,0.16}{##1}}}
\expandafter\def\csname PY@tok@sb\endcsname{\def\PY@tc##1{\textcolor[rgb]{0.73,0.13,0.13}{##1}}}
\expandafter\def\csname PY@tok@nt\endcsname{\let\PY@bf=\textbf\def\PY@tc##1{\textcolor[rgb]{0.00,0.50,0.00}{##1}}}
\expandafter\def\csname PY@tok@kn\endcsname{\let\PY@bf=\textbf\def\PY@tc##1{\textcolor[rgb]{0.00,0.50,0.00}{##1}}}
\expandafter\def\csname PY@tok@kc\endcsname{\let\PY@bf=\textbf\def\PY@tc##1{\textcolor[rgb]{0.00,0.50,0.00}{##1}}}
\expandafter\def\csname PY@tok@il\endcsname{\def\PY@tc##1{\textcolor[rgb]{0.40,0.40,0.40}{##1}}}
\expandafter\def\csname PY@tok@sh\endcsname{\def\PY@tc##1{\textcolor[rgb]{0.73,0.13,0.13}{##1}}}
\expandafter\def\csname PY@tok@sc\endcsname{\def\PY@tc##1{\textcolor[rgb]{0.73,0.13,0.13}{##1}}}
\expandafter\def\csname PY@tok@gt\endcsname{\def\PY@tc##1{\textcolor[rgb]{0.00,0.27,0.87}{##1}}}
\expandafter\def\csname PY@tok@o\endcsname{\def\PY@tc##1{\textcolor[rgb]{0.40,0.40,0.40}{##1}}}
\expandafter\def\csname PY@tok@mf\endcsname{\def\PY@tc##1{\textcolor[rgb]{0.40,0.40,0.40}{##1}}}
\expandafter\def\csname PY@tok@w\endcsname{\def\PY@tc##1{\textcolor[rgb]{0.73,0.73,0.73}{##1}}}
\expandafter\def\csname PY@tok@kt\endcsname{\def\PY@tc##1{\textcolor[rgb]{0.69,0.00,0.25}{##1}}}
\expandafter\def\csname PY@tok@mb\endcsname{\def\PY@tc##1{\textcolor[rgb]{0.40,0.40,0.40}{##1}}}
\expandafter\def\csname PY@tok@s\endcsname{\def\PY@tc##1{\textcolor[rgb]{0.73,0.13,0.13}{##1}}}
\expandafter\def\csname PY@tok@go\endcsname{\def\PY@tc##1{\textcolor[rgb]{0.53,0.53,0.53}{##1}}}
\expandafter\def\csname PY@tok@gh\endcsname{\let\PY@bf=\textbf\def\PY@tc##1{\textcolor[rgb]{0.00,0.00,0.50}{##1}}}
\expandafter\def\csname PY@tok@c\endcsname{\let\PY@it=\textit\def\PY@tc##1{\textcolor[rgb]{0.25,0.50,0.50}{##1}}}
\expandafter\def\csname PY@tok@nn\endcsname{\let\PY@bf=\textbf\def\PY@tc##1{\textcolor[rgb]{0.00,0.00,1.00}{##1}}}
\expandafter\def\csname PY@tok@mh\endcsname{\def\PY@tc##1{\textcolor[rgb]{0.40,0.40,0.40}{##1}}}
\expandafter\def\csname PY@tok@k\endcsname{\let\PY@bf=\textbf\def\PY@tc##1{\textcolor[rgb]{0.00,0.50,0.00}{##1}}}
\expandafter\def\csname PY@tok@mi\endcsname{\def\PY@tc##1{\textcolor[rgb]{0.40,0.40,0.40}{##1}}}
\expandafter\def\csname PY@tok@m\endcsname{\def\PY@tc##1{\textcolor[rgb]{0.40,0.40,0.40}{##1}}}
\expandafter\def\csname PY@tok@sd\endcsname{\let\PY@it=\textit\def\PY@tc##1{\textcolor[rgb]{0.73,0.13,0.13}{##1}}}
\expandafter\def\csname PY@tok@kp\endcsname{\def\PY@tc##1{\textcolor[rgb]{0.00,0.50,0.00}{##1}}}
\expandafter\def\csname PY@tok@gu\endcsname{\let\PY@bf=\textbf\def\PY@tc##1{\textcolor[rgb]{0.50,0.00,0.50}{##1}}}
\expandafter\def\csname PY@tok@cp\endcsname{\def\PY@tc##1{\textcolor[rgb]{0.74,0.48,0.00}{##1}}}
\expandafter\def\csname PY@tok@mo\endcsname{\def\PY@tc##1{\textcolor[rgb]{0.40,0.40,0.40}{##1}}}
\expandafter\def\csname PY@tok@gs\endcsname{\let\PY@bf=\textbf}
\expandafter\def\csname PY@tok@nl\endcsname{\def\PY@tc##1{\textcolor[rgb]{0.63,0.63,0.00}{##1}}}
\expandafter\def\csname PY@tok@nb\endcsname{\def\PY@tc##1{\textcolor[rgb]{0.00,0.50,0.00}{##1}}}
\expandafter\def\csname PY@tok@sr\endcsname{\def\PY@tc##1{\textcolor[rgb]{0.73,0.40,0.53}{##1}}}
\expandafter\def\csname PY@tok@cs\endcsname{\let\PY@it=\textit\def\PY@tc##1{\textcolor[rgb]{0.25,0.50,0.50}{##1}}}
\expandafter\def\csname PY@tok@bp\endcsname{\def\PY@tc##1{\textcolor[rgb]{0.00,0.50,0.00}{##1}}}
\expandafter\def\csname PY@tok@gd\endcsname{\def\PY@tc##1{\textcolor[rgb]{0.63,0.00,0.00}{##1}}}
\expandafter\def\csname PY@tok@si\endcsname{\let\PY@bf=\textbf\def\PY@tc##1{\textcolor[rgb]{0.73,0.40,0.53}{##1}}}
\expandafter\def\csname PY@tok@kd\endcsname{\let\PY@bf=\textbf\def\PY@tc##1{\textcolor[rgb]{0.00,0.50,0.00}{##1}}}
\expandafter\def\csname PY@tok@sx\endcsname{\def\PY@tc##1{\textcolor[rgb]{0.00,0.50,0.00}{##1}}}
\expandafter\def\csname PY@tok@gr\endcsname{\def\PY@tc##1{\textcolor[rgb]{1.00,0.00,0.00}{##1}}}
\expandafter\def\csname PY@tok@ge\endcsname{\let\PY@it=\textit}
\expandafter\def\csname PY@tok@gp\endcsname{\let\PY@bf=\textbf\def\PY@tc##1{\textcolor[rgb]{0.00,0.00,0.50}{##1}}}
\expandafter\def\csname PY@tok@se\endcsname{\let\PY@bf=\textbf\def\PY@tc##1{\textcolor[rgb]{0.73,0.40,0.13}{##1}}}
\expandafter\def\csname PY@tok@cm\endcsname{\let\PY@it=\textit\def\PY@tc##1{\textcolor[rgb]{0.25,0.50,0.50}{##1}}}
\expandafter\def\csname PY@tok@err\endcsname{\def\PY@bc##1{\setlength{\fboxsep}{0pt}\fcolorbox[rgb]{1.00,0.00,0.00}{1,1,1}{\strut ##1}}}

\def\PYZbs{\char`\\}
\def\PYZus{\char`\_}
\def\PYZob{\char`\{}
\def\PYZcb{\char`\}}
\def\PYZca{\char`\^}
\def\PYZam{\char`\&}
\def\PYZlt{\char`\<}
\def\PYZgt{\char`\>}
\def\PYZsh{\char`\#}
\def\PYZpc{\char`\%}
\def\PYZdl{\char`\$}
\def\PYZhy{\char`\-}
\def\PYZsq{\char`\'}
\def\PYZdq{\char`\"}
\def\PYZti{\char`\~}
% for compatibility with earlier versions
\def\PYZat{@}
\def\PYZlb{[}
\def\PYZrb{]}
\makeatother


    % Exact colors from NB
    \definecolor{incolor}{rgb}{0.0, 0.0, 0.5}
    \definecolor{outcolor}{rgb}{0.545, 0.0, 0.0}

    % Prevent overflowing lines due to hard-to-break entities
    \sloppy 
    % Slightly bigger margins than the latex defaults
 \hypersetup{pdfauthor={Yigal Weinstein},
            pdftitle={Data Science Notebook},
            pdfsubject={data science},
            pdfkeywords={data science, statistics, probability, machine learning},
            %pdfpagelayout={TwoPageLeft}, % Displays two pages, odd-numbered pages to the left 
            pdfcreator={PdfLaTeX},
            bookmarks={true},            %  A set of Acrobat bookmarks are written
            colorlinks={true},           %  Colors the text of links and anchors. 
            anchorcolor={black},         %  Color for anchor text.
            filecolor={cyan},            %  Color for URLs which open local files.
            menucolor={red},             %  Color for Acrobat menu items.
            runcolor={blue},              %  Color for run links (launch annotations).        
            linkcolor={red!50!black},
            citecolor={blue!50!black},
            urlcolor={blue!80!black}
            breaklinks=true,  % so long urls are correctly broken across lines
           }           
 
    \geometry{verbose,tmargin=1in,bmargin=1in,lmargin=1in,rmargin=1in}

\begin{document}
    
    
\maketitle

\section*{Introduction}
So the stage is set with the following description for the Stroop experiment:
\begin{quote}
In a Stroop task, participants are presented with a list of words, with
each word displayed in a color of ink. The participant's task is to say
out loud the color of the ink in which the word is printed. The task has
two conditions: a congruent words condition, and an incongruent words
condition. In the congruent words condition, the words being displayed
are color words whose names match the colors in which they are printed:
for example {\color{red} RED }, {\color{blue} BLUE }. In the incongruent words condition, the words
displayed are color words whose names do not match the colors in which
they are printed: for example { \color{yellow}PURPLE }, {\color{green} ORANGE }. In each case, we measure
the time it takes to name the ink colors in equally-sized lists. Each
participant will go through and record a time from each condition.
\cite{Udacity-P1-Stroop}
\end{quote}
\section{What is our independent variable? What is our dependent
variable?}\label{what-is-our-independent-variable-what-is-our-dependent-variable}

    The independent variable is a binary condition that of either congruency
or incongruency between the color of a word, and the symantic meaning of
the word. As example the word `RED' is a congruent example where the
word `red' has the color of its symantic meaning, i.e.~the color it
represents, and `BLUE' is an incongruent case where the word `blue'
which has the symantic meaning of the color blue is instead the color
green. Frome here on out the terms congruent color selection, CCS, and
incongruent color selection, ICS, will be used to denote the two
conditions the dependent variable may take that of a word representing a
color being colored with or with a different color respectively.

The dependent variable is the time, in seconds, it takes to read through
an equal sized lists of CCS or ICS words.

    \section{What is an appropriate set of hypotheses for this task? What
kind of statistical test do you expect to perform? Justify your
choices.}\label{what-is-an-appropriate-set-of-hypotheses-for-this-task-what-kind-of-statistical-test-do-you-expect-to-perform-justify-your-choices.}

    The null hypothesis, \(H_0\) is the statement that the ICS task will
take less or an equal amount of time to the CCS task. The alternative
hypothesis, \(H_A\) is the statement that the mean population time to
read an ICS list will be greater than an equally sized CCS list. The
null and alternative hypothesis choice stems from a need to test
quantatively the validity and extent of an already well respected
phenomenon the Stroop effect. The Stroop effect being defined to mean
that cognitive incongruences will lead to an increase in the reaction
time of a task.\\ It is true that this single tailed objective may not
provide insight if indeed the reaction time for ICS word lists take, on
the average, less time to complete, however this is a test regarding the
scope and veracity of the Stroop effect which has been shown in many
previous quantative tests to have general validity. If indeed the ICS
task takes less time it is enough that \(H_0\) should be kept to cause
doubt regarding the validity of either the test or the Stroop effect.
That is due to the large body of work already done to test the effect if
\(H_0\) is kept there is either something wrong with the experiment or
the Stroop effect has either more limitations on generalizability than
expected or there are indeed real issues with it's veracity. So to sum
things up via a set of equations:

\begin{align}
H_0:\; \mu_{ICS} &&\le&& \mu_{CCS} \\
H_A:\; \mu_{ICS} &&>&& \mu_{CCS}
\end{align}

    \section{Report some descriptive statistics regarding this dataset.
Include at least one measure of central tendency and at least one
measure of
variability.}\label{report-some-descriptive-statistics-regarding-this-dataset.-include-at-least-one-measure-of-central-tendency-and-at-least-one-measure-of-variability.}

    Below I've printed the mean, sample standard deviation, minimum values,
25th, 50th, 75th percentiles, maximums, and the median for both the CCS
and ICS samples, all values are in seconds. For central tendency I've
included both the mean, and the median - which in this case is simply
the 50th percentile. Notice the fact that for both CCS and ICS the
median and mean are nearly identical, for the CCS case it's less than
one third of a second difference, and for the ICS case the difference
between these two measure of central tendency is only slightly greater
than a second. This points to both distribution being evenly distributed
around the mean. As we can see for nearly all percentiles, minimum and
maximum, and mean values the CCS times are roughly two thirds of the
value of the ICS times, and all times for every category are
significantly longer for the ICS sample relative to the CCS.

An indication of the relative spread can be observed by looking at the
sample standard deviations. the ICS case is slightly more spread out
than the CCS case \(4.797s\) to \(3.559s\) respectively, i.e slightly
over a second more for the ICS sample. To gain a better descriptive
understanding of the spread we can also look at the range, i.e.
\(R = x_{max} - x_{min}\) of the two samples:

\begin{align}
R_{CCS} &= 22.328000 - 8.630000 & = 13.698\\
R_{ICS} &= 35.255000 - 15.687000 & = 19.568
\end{align}

From both the sample standard deviation and the range we can see that
the spread of the ICS sample is greater both in terms of the range,
i.e.~outliers, and of the sample standard deviation:

    \begin{Verbatim}[commandchars=\\\{\}]
{\color{incolor}In [{\color{incolor}2}]:} \PY{k+kn}{import} \PY{n+nn}{pandas} \PY{k}{as} \PY{n+nn}{pd}
        \PY{k+kn}{import} \PY{n+nn}{numpy} \PY{k}{as} \PY{n+nn}{np}
        \PY{n}{stroopdata} \PY{o}{=} \PY{n}{pd}\PY{o}{.}\PY{n}{read\PYZus{}csv}\PY{p}{(}\PY{l+s}{\PYZsq{}}\PY{l+s}{stroopdata.csv}\PY{l+s}{\PYZsq{}}\PY{p}{)}
        \PY{n}{stroopdata}\PY{o}{.}\PY{n}{describe}\PY{p}{(}\PY{p}{)}
\end{Verbatim}

            \begin{Verbatim}[commandchars=\\\{\}]
{\color{outcolor}Out[{\color{outcolor}2}]:}        Congruent  Incongruent
        count  24.000000    24.000000
        mean   14.051125    22.015917
        std     3.559358     4.797057
        min     8.630000    15.687000
        25\%    11.895250    18.716750
        50\%    14.356500    21.017500
        75\%    16.200750    24.051500
        max    22.328000    35.255000
\end{Verbatim}
        
    and a simple calculation of the range for the CCS and ICS samples:

    \begin{Verbatim}[commandchars=\\\{\}]
{\color{incolor}In [{\color{incolor}3}]:} \PY{n+nb}{print}\PY{p}{(}\PY{l+s}{\PYZdq{}}\PY{l+s+se}{\PYZbs{}n}\PY{l+s}{The range for the two cases:}\PY{l+s+se}{\PYZbs{}n}\PY{l+s}{\PYZdq{}}\PY{p}{,} \PY{n}{stroopdata}\PY{o}{.}\PY{n}{max}\PY{p}{(}\PY{p}{)} \PY{o}{\PYZhy{}} \PY{n}{stroopdata}\PY{o}{.}\PY{n}{min}\PY{p}{(}\PY{p}{)}\PY{p}{)}
\end{Verbatim}

    \begin{Verbatim}[commandchars=\\\{\}]
The range for the two cases:
 Congruent      13.698
Incongruent    19.568
dtype: float64
    \end{Verbatim}

    \section{Provide one or two visualizations that show the distribution of
the sample data. Write one or two sentences noting what you observe
about the plot or
plots.}\label{provide-one-or-two-visualizations-that-show-the-distribution-of-the-sample-data.-write-one-or-two-sentences-noting-what-you-observe-about-the-plot-or-plots.}

    

    \begin{Verbatim}[commandchars=\\\{\}]
{\color{incolor}In [{\color{incolor}17}]:} \PY{k+kn}{import} \PY{n+nn}{numpy} \PY{k}{as} \PY{n+nn}{np}
         \PY{k+kn}{import} \PY{n+nn}{matplotlib}\PY{n+nn}{.}\PY{n+nn}{pyplot} \PY{k}{as} \PY{n+nn}{plt}
         
         \PY{o}{\PYZpc{}}\PY{k}{pylab} inline
         \PY{n}{pylab}\PY{o}{.}\PY{n}{rcParams}\PY{p}{[}\PY{l+s}{\PYZsq{}}\PY{l+s}{figure.figsize}\PY{l+s}{\PYZsq{}}\PY{p}{]} \PY{o}{=} \PY{p}{(}\PY{l+m+mf}{8.0}\PY{p}{,} \PY{l+m+mf}{7.0}\PY{p}{)}
         
         \PY{n}{fig}\PY{p}{,} \PY{n}{ax} \PY{o}{=} \PY{n}{plt}\PY{o}{.}\PY{n}{subplots}\PY{p}{(}\PY{p}{)}
         
         \PY{n}{index} \PY{o}{=} \PY{n}{np}\PY{o}{.}\PY{n}{arange}\PY{p}{(}\PY{n}{stroopdata}\PY{o}{.}\PY{n}{shape}\PY{p}{[}\PY{l+m+mi}{0}\PY{p}{]}\PY{p}{)}
         \PY{n}{bar\PYZus{}width} \PY{o}{=} \PY{l+m+mf}{0.36}
         
         \PY{n}{opacity} \PY{o}{=} \PY{l+m+mf}{0.6}
         
         \PY{n}{rects1} \PY{o}{=} \PY{n}{plt}\PY{o}{.}\PY{n}{bar}\PY{p}{(}\PY{n}{index}\PY{p}{,} \PY{n}{stroopdata}\PY{p}{[}\PY{l+s}{\PYZsq{}}\PY{l+s}{Congruent}\PY{l+s}{\PYZsq{}}\PY{p}{]}\PY{p}{,} 
                          \PY{n}{bar\PYZus{}width}\PY{p}{,}
                          \PY{n}{alpha}\PY{o}{=}\PY{n}{opacity}\PY{p}{,}
                          \PY{n}{color}\PY{o}{=}\PY{l+s}{\PYZsq{}}\PY{l+s}{k}\PY{l+s}{\PYZsq{}}\PY{p}{,}
                          \PY{n}{label}\PY{o}{=}\PY{l+s}{\PYZsq{}}\PY{l+s}{CCS Sample}\PY{l+s}{\PYZsq{}}\PY{p}{)}
         
         \PY{n}{rects2} \PY{o}{=} \PY{n}{plt}\PY{o}{.}\PY{n}{bar}\PY{p}{(}\PY{n}{index} \PY{o}{+} \PY{n}{bar\PYZus{}width}\PY{p}{,} \PY{n}{stroopdata}\PY{p}{[}\PY{l+s}{\PYZsq{}}\PY{l+s}{Incongruent}\PY{l+s}{\PYZsq{}}\PY{p}{]}\PY{p}{,} 
                          \PY{n}{bar\PYZus{}width}\PY{p}{,}
                          \PY{n}{alpha}\PY{o}{=}\PY{n}{opacity}\PY{p}{,}
                          \PY{n}{color}\PY{o}{=}\PY{l+s}{\PYZsq{}}\PY{l+s}{b}\PY{l+s}{\PYZsq{}}\PY{p}{,}
                          \PY{n}{label}\PY{o}{=}\PY{l+s}{\PYZsq{}}\PY{l+s}{ICS Sample}\PY{l+s}{\PYZsq{}}\PY{p}{)}
         
         \PY{c}{\PYZsh{} add some text for labels, title and axes ticks}
         \PY{n}{plt}\PY{o}{.}\PY{n}{rc}\PY{p}{(}\PY{l+s}{\PYZsq{}}\PY{l+s}{text}\PY{l+s}{\PYZsq{}}\PY{p}{,} \PY{n}{usetex}\PY{o}{=}\PY{k}{True}\PY{p}{)}
         \PY{n}{plt}\PY{o}{.}\PY{n}{rc}\PY{p}{(}\PY{l+s}{\PYZsq{}}\PY{l+s}{font}\PY{l+s}{\PYZsq{}}\PY{p}{,} \PY{n}{family}\PY{o}{=}\PY{l+s}{\PYZsq{}}\PY{l+s}{sans}\PY{l+s}{\PYZsq{}}\PY{p}{)}
         \PY{n}{plt}\PY{o}{.}\PY{n}{xlabel}\PY{p}{(}\PY{l+s}{r\PYZsq{}}\PY{l+s}{\PYZbs{}}\PY{l+s}{textbf\PYZob{}Subject Number\PYZcb{}}\PY{l+s}{\PYZsq{}}\PY{p}{)}
         \PY{n}{plt}\PY{o}{.}\PY{n}{ylabel}\PY{p}{(}\PY{l+s}{r\PYZsq{}}\PY{l+s}{\PYZbs{}}\PY{l+s}{textit\PYZob{}Time\PYZcb{} (s)}\PY{l+s}{\PYZsq{}}\PY{p}{,} \PY{n}{fontsize}\PY{o}{=}\PY{l+m+mi}{16}\PY{p}{)}
         \PY{n}{plt}\PY{o}{.}\PY{n}{title}\PY{p}{(}\PY{l+s}{r\PYZsq{}}\PY{l+s}{Scores for CCS and ICS}\PY{l+s}{\PYZsq{}}\PY{p}{,}\PY{n}{fontsize}\PY{o}{=}\PY{l+m+mi}{16}\PY{p}{,} \PY{n}{color}\PY{o}{=}\PY{l+s}{\PYZsq{}}\PY{l+s}{gray}\PY{l+s}{\PYZsq{}}\PY{p}{)}
         \PY{n}{plt}\PY{o}{.}\PY{n}{xticks}\PY{p}{(}\PY{n}{index} \PY{o}{+} \PY{n}{bar\PYZus{}width}\PY{p}{,} \PY{n+nb}{range}\PY{p}{(}\PY{l+m+mi}{1}\PY{p}{,}\PY{l+m+mi}{25}\PY{p}{)}\PY{p}{)}
         \PY{n}{plt}\PY{o}{.}\PY{n}{legend}\PY{p}{(}\PY{p}{)}
         
         \PY{n}{plt}\PY{o}{.}\PY{n}{tight\PYZus{}layout}\PY{p}{(}\PY{p}{)}
         \PY{n}{plt}\PY{o}{.}\PY{n}{show}\PY{p}{(}\PY{p}{)}
\end{Verbatim}

    \begin{Verbatim}[commandchars=\\\{\}]
Populating the interactive namespace from numpy and matplotlib
    \end{Verbatim}

    \begin{center}
    \adjustimage{max size={0.9\linewidth}{0.9\paperheight}}{Stroop Effect_files/Stroop Effect_13_1.png}
    \end{center}
    { \hspace*{\fill} \\}
    
    \begin{Verbatim}[commandchars=\\\{\}]
{\color{incolor}In [{\color{incolor}12}]:} \PY{l+s+sd}{\PYZdq{}\PYZdq{}\PYZdq{}}
         \PY{l+s+sd}{Demo of TeX rendering.}
         
         \PY{l+s+sd}{You can use TeX to render all of your matplotlib text if the rc}
         \PY{l+s+sd}{parameter text.usetex is set.  This works currently on the agg and ps}
         \PY{l+s+sd}{backends, and requires that you have tex and the other dependencies}
         \PY{l+s+sd}{described at http://matplotlib.org/users/usetex.html}
         \PY{l+s+sd}{properly installed on your system.  The first time you run a script}
         \PY{l+s+sd}{you will see a lot of output from tex and associated tools.  The next}
         \PY{l+s+sd}{time, the run may be silent, as a lot of the information is cached in}
         \PY{l+s+sd}{\PYZti{}/.tex.cache}
         
         \PY{l+s+sd}{\PYZdq{}\PYZdq{}\PYZdq{}}
         \PY{k+kn}{import} \PY{n+nn}{numpy} \PY{k}{as} \PY{n+nn}{np}
         \PY{k+kn}{import} \PY{n+nn}{matplotlib}\PY{n+nn}{.}\PY{n+nn}{pyplot} \PY{k}{as} \PY{n+nn}{plt}
         
         
         \PY{c}{\PYZsh{} Example data}
         \PY{n}{t} \PY{o}{=} \PY{n}{np}\PY{o}{.}\PY{n}{arange}\PY{p}{(}\PY{l+m+mf}{0.0}\PY{p}{,} \PY{l+m+mf}{1.0} \PY{o}{+} \PY{l+m+mf}{0.01}\PY{p}{,} \PY{l+m+mf}{0.01}\PY{p}{)}
         \PY{n}{s} \PY{o}{=} \PY{n}{np}\PY{o}{.}\PY{n}{cos}\PY{p}{(}\PY{l+m+mi}{4} \PY{o}{*} \PY{n}{np}\PY{o}{.}\PY{n}{pi} \PY{o}{*} \PY{n}{t}\PY{p}{)} \PY{o}{+} \PY{l+m+mi}{2}
         
         \PY{n}{plt}\PY{o}{.}\PY{n}{rc}\PY{p}{(}\PY{l+s}{\PYZsq{}}\PY{l+s}{text}\PY{l+s}{\PYZsq{}}\PY{p}{,} \PY{n}{usetex}\PY{o}{=}\PY{k}{True}\PY{p}{)}
         \PY{n}{plt}\PY{o}{.}\PY{n}{rc}\PY{p}{(}\PY{l+s}{\PYZsq{}}\PY{l+s}{font}\PY{l+s}{\PYZsq{}}\PY{p}{,} \PY{n}{family}\PY{o}{=}\PY{l+s}{\PYZsq{}}\PY{l+s}{sans}\PY{l+s}{\PYZsq{}}\PY{p}{)}
         \PY{n}{plt}\PY{o}{.}\PY{n}{plot}\PY{p}{(}\PY{n}{t}\PY{p}{,} \PY{n}{s}\PY{p}{)}
         
         \PY{n}{plt}\PY{o}{.}\PY{n}{xlabel}\PY{p}{(}\PY{l+s}{r\PYZsq{}}\PY{l+s}{\PYZbs{}}\PY{l+s}{textbf\PYZob{}time\PYZcb{} (s)}\PY{l+s}{\PYZsq{}}\PY{p}{)}
         \PY{n}{plt}\PY{o}{.}\PY{n}{ylabel}\PY{p}{(}\PY{l+s}{r\PYZsq{}}\PY{l+s}{\PYZbs{}}\PY{l+s}{textit\PYZob{}voltage\PYZcb{} (mV)}\PY{l+s}{\PYZsq{}}\PY{p}{,}\PY{n}{fontsize}\PY{o}{=}\PY{l+m+mi}{16}\PY{p}{)}
         \PY{n}{plt}\PY{o}{.}\PY{n}{title}\PY{p}{(}\PY{l+s}{r\PYZdq{}}\PY{l+s}{\PYZbs{}}\PY{l+s}{TeX}\PY{l+s}{\PYZbs{}}\PY{l+s}{ is Number }\PY{l+s}{\PYZdq{}}
                   \PY{l+s}{r\PYZdq{}}\PY{l+s}{\PYZdl{}}\PY{l+s}{\PYZbs{}}\PY{l+s}{displaystyle}\PY{l+s}{\PYZbs{}}\PY{l+s}{sum\PYZus{}\PYZob{}n=1\PYZcb{}\PYZca{}}\PY{l+s}{\PYZbs{}}\PY{l+s}{infty}\PY{l+s}{\PYZbs{}}\PY{l+s}{frac\PYZob{}\PYZhy{}e\PYZca{}\PYZob{}i}\PY{l+s}{\PYZbs{}}\PY{l+s}{pi\PYZcb{}\PYZcb{}\PYZob{}2\PYZca{}n\PYZcb{}\PYZdl{}!}\PY{l+s}{\PYZdq{}}\PY{p}{,}
                   \PY{n}{fontsize}\PY{o}{=}\PY{l+m+mi}{16}\PY{p}{,} \PY{n}{color}\PY{o}{=}\PY{l+s}{\PYZsq{}}\PY{l+s}{gray}\PY{l+s}{\PYZsq{}}\PY{p}{)}
         \PY{c}{\PYZsh{} Make room for the ridiculously large title.}
         \PY{n}{plt}\PY{o}{.}\PY{n}{subplots\PYZus{}adjust}\PY{p}{(}\PY{n}{top}\PY{o}{=}\PY{l+m+mf}{0.8}\PY{p}{)}
         
         \PY{n}{plt}\PY{o}{.}\PY{n}{savefig}\PY{p}{(}\PY{l+s}{\PYZsq{}}\PY{l+s}{tex\PYZus{}demo}\PY{l+s}{\PYZsq{}}\PY{p}{)}
         \PY{n}{plt}\PY{o}{.}\PY{n}{show}\PY{p}{(}\PY{p}{)}
\end{Verbatim}

    \begin{center}
    \adjustimage{max size={0.9\linewidth}{0.9\paperheight}}{Stroop Effect_files/Stroop Effect_14_0.png}
    \end{center}
    { \hspace*{\fill} \\}
    
    The bar graph above shows graphically that each subject took more time
on the ICS word list than on the CCS word list.

    \section{Now, perform the statistical test and report your results. What
is your confidence level and your critical statistic value? Do you
reject the null hypothesis or fail to reject it? Come to a conclusion in
terms of the experiment task. Did the results match up with your
expectations?}\label{now-perform-the-statistical-test-and-report-your-results.-what-is-your-confidence-level-and-your-critical-statistic-value-do-you-reject-the-null-hypothesis-or-fail-to-reject-it-come-to-a-conclusion-in-terms-of-the-experiment-task.-did-the-results-match-up-with-your-expectations}

    \section{What do you think is responsible for the effects observed? Can
you think of an alternative or similar task that would result in a
similar effect? Some research about the problem will be helpful for
thinking about these two
questions!}\label{what-do-you-think-is-responsible-for-the-effects-observed-can-you-think-of-an-alternative-or-similar-task-that-would-result-in-a-similar-effect-some-research-about-the-problem-will-be-helpful-for-thinking-about-these-two-questions}

    \paragraph{Refrences:}\label{refrences}

    \begin{itemize}
\item
  PANDAS:
\item
  http://pandas.pydata.org/pandas-docs/stable/io.html\#io-read-csv-table
  \# link for reading in CSV file
\item
  Udacity:
\item
  https://docs.google.com/document/d/1-OkpZLjG\_kX9J6LIQ5IltsqMzVWjh36QpnP2RYpVdPU/pub
  \# Udacity official document on general instructions for further
  projects as well as specifically for the Stroop Effect, P1.
\item
  https://docs.google.com/document/d/1bqyi1Fm5truesLhmbAq16Zl-Ajj9bnNIU\_68P60nDQg/pub
  \# Udacity official documentation outlining what is expected to
  satisfactorily complete a project
\item
  matplotlib
\item
  http://matplotlib.org/examples/api/barchart\_demo.html \# The bar
  graph example was used as a basis for the bar graph in this report.
\item
  http://matplotlib.org/api/colors\_api.html \# understanding colors in
  matplotlib
\item
  http://matplotlib.org/users/usetex.html \# using LaTeX in matplotlib
  graphs
\end{itemize}

    \begin{Verbatim}[commandchars=\\\{\}]
{\color{incolor}In [{\color{incolor}19}]:} \PY{o}{\PYZpc{}\PYZpc{}}\PY{k}{bash}
         ipython3 nbconvert \PYZhy{}\PYZhy{}template udacity.tplx  \PYZhy{}\PYZhy{}to latex \PYZdq{}Stroop Effect.ipynb\PYZdq{} 2\PYZgt{}\PYZam{}1 \PYZgt{} build.log
         pdflatex \PYZdq{}Stroop Effect\PYZdq{} \PYZgt{}\PYZgt{} build.log
         bibtex \PYZdq{}Stroop Effect\PYZdq{} \PYZgt{}\PYZgt{} build.log
         pdflatex \PYZdq{}Stroop Effect\PYZdq{} \PYZgt{}\PYZgt{} build.log
\end{Verbatim}

    \begin{Verbatim}[commandchars=\\\{\}]
/home/yigal/.local/lib/python3.4/site-packages/IPython/nbconvert.py:13: ShimWarning: The `IPython.nbconvert` package has been deprecated. You should import from ipython\_nbconvert instead.
  "You should import from ipython\_nbconvert instead.", ShimWarning)
[NbConvertApp] Converting notebook Stroop Effect.ipynb to latex
[NbConvertApp] Support files will be in Stroop Effect\_files/
[NbConvertApp] Making directory Stroop Effect\_files
[NbConvertApp] Making directory Stroop Effect\_files
[NbConvertApp] Writing 32871 bytes to Stroop Effect.tex
    \end{Verbatim}

    \begin{Verbatim}[commandchars=\\\{\}]
{\color{incolor}In [{\color{incolor} }]:} \PY{o}{\PYZpc{}}\PY{k}{ls}
\end{Verbatim}

    \begin{Verbatim}[commandchars=\\\{\}]
{\color{incolor}In [{\color{incolor} }]:} \PY{c}{\PYZsh{}from notebook.nbextensions import install\PYZus{}nbextension}
        \PY{c}{\PYZsh{}install\PYZus{}nbextension(\PYZsq{}https://goo.gl/5TK96v\PYZsq{}, user=True, destination=\PYZdq{}vim\PYZus{}binding.js\PYZdq{})}
        \PY{c}{\PYZsh{} Or if you prefre a full URL}
        \PY{c}{\PYZsh{}install\PYZus{}nbextension(\PYZsq{}https://rawgithub.com/lambdalisue/jupyter\PYZhy{}vim\PYZhy{}binding/master/nbextensions/vim\PYZus{}binding.js\PYZsq{}, user=True)}
        \PY{c}{\PYZsh{}Jupyter.utils.load\PYZus{}extensions(\PYZsq{}vim\PYZus{}binding\PYZsq{})}
\end{Verbatim}

    \begin{Verbatim}[commandchars=\\\{\}]
{\color{incolor}In [{\color{incolor}8}]:} \PY{o}{\PYZpc{}\PYZpc{}}\PY{k}{javascript}
        require([\PYZsq{}base/js/utils\PYZsq{}],
        function(utils) \PYZob{}
           utils.load\PYZus{}extensions(\PYZsq{}calico\PYZhy{}spell\PYZhy{}check\PYZsq{}, \PYZsq{}calico\PYZhy{}document\PYZhy{}tools\PYZsq{}, \PYZsq{}calico\PYZhy{}cell\PYZhy{}tools\PYZsq{});
        \PYZcb{});
\end{Verbatim}

    
    \begin{verbatim}
<IPython.core.display.Javascript object>
    \end{verbatim}

    
    Examples of citations: \hyperref[cite-PER-GRA:2007]{(Pérez and Granger,
2007)} or \hyperref[cite-Papa2007]{(Papa and Markov, 2007)}
\hyperref[cite-Udacity2015]{(Udacity, 2015)} \hyperref[cite-WinNT]{(,
1999)}.

    \begin{Verbatim}[commandchars=\\\{\}]
{\color{incolor}In [{\color{incolor}24}]:} \PY{o}{!}gksu ipython3 install\PYZhy{}nbextension https://bitbucket.org/ipre/calico/downloads/calico\PYZhy{}document\PYZhy{}tools\PYZhy{}1.0.zip
\end{Verbatim}

    \begin{Verbatim}[commandchars=\\\{\}]
(gksu:11779): GConf-CRITICAL **: gconf\_value\_free: assertion 'value != NULL' failed
Traceback (most recent call last):
  File "/usr/local/bin/ipython3", line 11, in <module>
    sys.exit(start\_ipython())
  File "/usr/local/lib/python3.4/dist-packages/IPython/\_\_init\_\_.py", line 118, in start\_ipython
    return launch\_new\_instance(argv=argv, **kwargs)
  File "/usr/local/lib/python3.4/dist-packages/traitlets/config/application.py", line 591, in launch\_instance
    app.initialize(argv)
  File "<decorator-gen-111>", line 2, in initialize
  File "/usr/local/lib/python3.4/dist-packages/traitlets/config/application.py", line 75, in catch\_config\_error
    return method(app, *args, **kwargs)
  File "/usr/local/lib/python3.4/dist-packages/IPython/terminal/ipapp.py", line 305, in initialize
    super(TerminalIPythonApp, self).initialize(argv)
  File "<decorator-gen-7>", line 2, in initialize
  File "/usr/local/lib/python3.4/dist-packages/traitlets/config/application.py", line 75, in catch\_config\_error
    return method(app, *args, **kwargs)
  File "/usr/local/lib/python3.4/dist-packages/IPython/core/application.py", line 386, in initialize
    self.parse\_command\_line(argv)
  File "/usr/local/lib/python3.4/dist-packages/IPython/terminal/ipapp.py", line 300, in parse\_command\_line
    return super(TerminalIPythonApp, self).parse\_command\_line(argv)
  File "<decorator-gen-4>", line 2, in parse\_command\_line
  File "/usr/local/lib/python3.4/dist-packages/traitlets/config/application.py", line 75, in catch\_config\_error
    return method(app, *args, **kwargs)
  File "/usr/local/lib/python3.4/dist-packages/traitlets/config/application.py", line 487, in parse\_command\_line
    return self.initialize\_subcommand(subc, subargv)
  File "<decorator-gen-3>", line 2, in initialize\_subcommand
  File "/usr/local/lib/python3.4/dist-packages/traitlets/config/application.py", line 75, in catch\_config\_error
    return method(app, *args, **kwargs)
  File "/usr/local/lib/python3.4/dist-packages/traitlets/config/application.py", line 418, in initialize\_subcommand
    subapp = import\_item(subapp)
  File "/usr/local/lib/python3.4/dist-packages/ipython\_genutils/importstring.py", line 31, in import\_item
    module = \_\_import\_\_(package, fromlist=[obj])
ImportError: No module named 'notebook'
    \end{Verbatim}

    \begin{Verbatim}[commandchars=\\\{\}]
{\color{incolor}In [{\color{incolor}20}]:} \PY{n}{IPython}\PY{o}{.}\PY{n}{display}\PY{o}{.}\PY{n}{YouTubeVideo}\PY{p}{(}\PY{l+s}{\PYZdq{}}\PY{l+s}{YbM8rrj\PYZhy{}Bms}\PY{l+s}{\PYZdq{}}\PY{p}{)}
\end{Verbatim}

    \begin{Verbatim}[commandchars=\\\{\}]

        ---------------------------------------------------------------------------

        NameError                                 Traceback (most recent call last)

        <ipython-input-20-8f1b1a3b6f91> in <module>()
    ----> 1 IPython.display.YouTubeVideo("YbM8rrj-Bms")
    

        NameError: name 'IPython' is not defined

    \end{Verbatim}

%{[}\^{}{]}(\#ref-2) Pérez, Fernando and Granger, Brian E.. 2007.
%\emph{IPython: a System for Interactive Scientific Computing}.
%\href{http://ipython.org}{URL}
%
%{[}\^{}{]}(\#ref-3) Papa, David A. and Markov, Igor L.. 2007.
%\emph{Hypergraph partitioning and clustering}.
%\href{http://www.podload.org/pubs/book/part_survey.pdf}{URL}
%
%{[}\^{}{]}(\#ref-4) Udacity, Udacity. 2015. \emph{Statistics: The
%Science of Decisions - Project Instructions}.
%\href{https://docs.google.com/document/d/1-OkpZLjG_kX9J6LIQ5IltsqMzVWjh36QpnP2RYpVdPU/pub}{URL}
%
%{[}\^{}{]}(\#ref-5) MultiMedia LLC. 1999. \emph{MS Windows NT Kernel
%Description}.
%\href{http://web.archive.org/web/20080207010024/http://www.808multimedia.com/winnt/kernel.htm}{URL}

\bibliographystyle{unsrt}
\bibliography{udacity}
\end{document}
