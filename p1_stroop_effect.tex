\documentclass{article}
    \usepackage{graphicx} % Used to insert images
    \usepackage{adjustbox} % Used to constrain images to a maximum size 
    \usepackage{xcolor} % Allow colors to be defined
    \usepackage{enumerate} % Needed for markdown enumerations to work
    \usepackage{geometry} % Used to adjust the document margins
    \usepackage{amsmath} % Equations
    \usepackage{amssymb} % Equations
    %\usepackage[mathletters]{ucs} % Extended unicode (utf-8) support
    \usepackage[T1]{fontenc}
    \usepackage{newtxtext,newtxmath}
    \usepackage[utf8]{inputenc} % Allow utf-8 characters in the tex document
    \usepackage{fancyvrb} % verbatim replacement that allows latex
    \usepackage{grffile} % extends the file name processing of package graphics 
                         % to support a larger range 
    % The hyperref package gives us a pdf with properly built
    % internal navigation ('pdf bookmarks' for the table of contents,
    % internal cross-reference links, web links for URLs, etc.)
    \usepackage{tikz}
    \usepackage{pgfplots}
    \usepackage[]{ulem}
    \pgfplotsset{compat=newest}
    \usepackage{url}
    \usepackage{longtable} % longtable support required by pandoc >1.10
    \usepackage{booktabs}  % table support for pandoc > 1.12.2
    \usepackage{ulem} % ulem is needed to support strikethroughs (\sout)
    \usepackage{titlesec}
    \titleformat{\section}{\normalfont\Large\bfseries}{Question {\thesection}: }{1em}{}
    \usepackage[
      autocite=inline, 
      backend=biber,
      labeldate=true, 
      refsegment=section,
      uniquename=full,
      uniquelist=true]
    {biblatex}
    \usepackage{hyperref}
    \addbibresource{udacity.bib}
    \definecolor{orange}{cmyk}{0,0.4,0.8,0.2}
    \definecolor{darkorange}{rgb}{.71,0.21,0.01}
    \definecolor{darkgreen}{rgb}{.12,.54,.11}
    \definecolor{myteal}{rgb}{.26, .44, .56}
    \definecolor{gray}{gray}{0.45}
    \definecolor{lightgray}{gray}{.95}
    \definecolor{mediumgray}{gray}{.8}
    \definecolor{inputbackground}{rgb}{.95, .95, .85}
    \definecolor{outputbackground}{rgb}{.95, .95, .95}
    \definecolor{traceback}{rgb}{1, .95, .95}
    % ansi colors
    \definecolor{red}{rgb}{.6,0,0}
    \definecolor{green}{rgb}{0,.65,0}
    \definecolor{brown}{rgb}{0.6,0.6,0}
    \definecolor{blue}{rgb}{0,.145,.698}
    \definecolor{purple}{rgb}{.698,.145,.698}
    \definecolor{cyan}{rgb}{0,.698,.698}
    \definecolor{lightgray}{gray}{0.5}
    
    % bright ansi colors
    \definecolor{darkgray}{gray}{0.25}
    \definecolor{lightred}{rgb}{1.0,0.39,0.28}
    \definecolor{lightgreen}{rgb}{0.48,0.99,0.0}
    \definecolor{lightblue}{rgb}{0.53,0.81,0.92}
    \definecolor{lightpurple}{rgb}{0.87,0.63,0.87}
    \definecolor{lightcyan}{rgb}{0.5,1.0,0.83}
    
    % commands and environments needed by pandoc snippets
    % extracted from the output of `pandoc -s`
    \providecommand{\tightlist}{%
      \setlength{\itemsep}{0pt}\setlength{\parskip}{0pt}}
    \DefineVerbatimEnvironment{Highlighting}{Verbatim}{commandchars=\\\{\}}
    % Add ',fontsize=\small' for more characters per line
    \newenvironment{Shaded}{}{}
    \newcommand{\KeywordTok}[1]{\textcolor[rgb]{0.00,0.44,0.13}{\textbf{{#1}}}}
    \newcommand{\DataTypeTok}[1]{\textcolor[rgb]{0.56,0.13,0.00}{{#1}}}
    \newcommand{\DecValTok}[1]{\textcolor[rgb]{0.25,0.63,0.44}{{#1}}}
    \newcommand{\BaseNTok}[1]{\textcolor[rgb]{0.25,0.63,0.44}{{#1}}}
    \newcommand{\FloatTok}[1]{\textcolor[rgb]{0.25,0.63,0.44}{{#1}}}
    \newcommand{\CharTok}[1]{\textcolor[rgb]{0.25,0.44,0.63}{{#1}}}
    \newcommand{\StringTok}[1]{\textcolor[rgb]{0.25,0.44,0.63}{{#1}}}
    \newcommand{\CommentTok}[1]{\textcolor[rgb]{0.38,0.63,0.69}{\textit{{#1}}}}
    \newcommand{\OtherTok}[1]{\textcolor[rgb]{0.00,0.44,0.13}{{#1}}}
    \newcommand{\AlertTok}[1]{\textcolor[rgb]{1.00,0.00,0.00}{\textbf{{#1}}}}
    \newcommand{\FunctionTok}[1]{\textcolor[rgb]{0.02,0.16,0.49}{{#1}}}
    \newcommand{\RegionMarkerTok}[1]{{#1}}
    \newcommand{\ErrorTok}[1]{\textcolor[rgb]{1.00,0.00,0.00}{\textbf{{#1}}}}
    \newcommand{\NormalTok}[1]{{#1}}
    
    % Additional commands for more recent versions of Pandoc
    \newcommand{\ConstantTok}[1]{\textcolor[rgb]{0.53,0.00,0.00}{{#1}}}
    \newcommand{\SpecialCharTok}[1]{\textcolor[rgb]{0.25,0.44,0.63}{{#1}}}
    \newcommand{\VerbatimStringTok}[1]{\textcolor[rgb]{0.25,0.44,0.63}{{#1}}}
    \newcommand{\SpecialStringTok}[1]{\textcolor[rgb]{0.73,0.40,0.53}{{#1}}}
    \newcommand{\ImportTok}[1]{{#1}}
    \newcommand{\DocumentationTok}[1]{\textcolor[rgb]{0.73,0.13,0.13}{\textit{{#1}}}}
    \newcommand{\AnnotationTok}[1]{\textcolor[rgb]{0.38,0.63,0.69}{\textbf{\textit{{#1}}}}}
    \newcommand{\CommentVarTok}[1]{\textcolor[rgb]{0.38,0.63,0.69}{\textbf{\textit{{#1}}}}}
    \newcommand{\VariableTok}[1]{\textcolor[rgb]{0.10,0.09,0.49}{{#1}}}
    \newcommand{\ControlFlowTok}[1]{\textcolor[rgb]{0.00,0.44,0.13}{\textbf{{#1}}}}
    \newcommand{\OperatorTok}[1]{\textcolor[rgb]{0.40,0.40,0.40}{{#1}}}
    \newcommand{\BuiltInTok}[1]{{#1}}
    \newcommand{\ExtensionTok}[1]{{#1}}
    \newcommand{\PreprocessorTok}[1]{\textcolor[rgb]{0.74,0.48,0.00}{{#1}}}
    \newcommand{\AttributeTok}[1]{\textcolor[rgb]{0.49,0.56,0.16}{{#1}}}
    \newcommand{\InformationTok}[1]{\textcolor[rgb]{0.38,0.63,0.69}{\textbf{\textit{{#1}}}}}
    \newcommand{\WarningTok}[1]{\textcolor[rgb]{0.38,0.63,0.69}{\textbf{\textit{{#1}}}}}
    
    
    % Define a nice break command that doesn't care if a line doesn't already
    % exist.
    \def\br{\hspace*{\fill} \\* }
    % Math Jax compatability definitions
    \def\gt{>}
    \def\lt{<}
    % Document parameters
    
\title{Exploring the Stroop Effect using Python}
\author{Yigal Weinstein}

    % Pygments definitions
    
\makeatletter
\def\PY@reset{\let\PY@it=\relax \let\PY@bf=\relax%
    \let\PY@ul=\relax \let\PY@tc=\relax%
    \let\PY@bc=\relax \let\PY@ff=\relax}
\def\PY@tok#1{\csname PY@tok@#1\endcsname}
\def\PY@toks#1+{\ifx\relax#1\empty\else%
    \PY@tok{#1}\expandafter\PY@toks\fi}
\def\PY@do#1{\PY@bc{\PY@tc{\PY@ul{%
    \PY@it{\PY@bf{\PY@ff{#1}}}}}}}
\def\PY#1#2{\PY@reset\PY@toks#1+\relax+\PY@do{#2}}

\expandafter\def\csname PY@tok@ni\endcsname{\let\PY@bf=\textbf\def\PY@tc##1{\textcolor[rgb]{0.60,0.60,0.60}{##1}}}
\expandafter\def\csname PY@tok@gi\endcsname{\def\PY@tc##1{\textcolor[rgb]{0.00,0.63,0.00}{##1}}}
\expandafter\def\csname PY@tok@nd\endcsname{\def\PY@tc##1{\textcolor[rgb]{0.67,0.13,1.00}{##1}}}
\expandafter\def\csname PY@tok@nv\endcsname{\def\PY@tc##1{\textcolor[rgb]{0.10,0.09,0.49}{##1}}}
\expandafter\def\csname PY@tok@s2\endcsname{\def\PY@tc##1{\textcolor[rgb]{0.73,0.13,0.13}{##1}}}
\expandafter\def\csname PY@tok@ne\endcsname{\let\PY@bf=\textbf\def\PY@tc##1{\textcolor[rgb]{0.82,0.25,0.23}{##1}}}
\expandafter\def\csname PY@tok@ss\endcsname{\def\PY@tc##1{\textcolor[rgb]{0.10,0.09,0.49}{##1}}}
\expandafter\def\csname PY@tok@vi\endcsname{\def\PY@tc##1{\textcolor[rgb]{0.10,0.09,0.49}{##1}}}
\expandafter\def\csname PY@tok@c1\endcsname{\let\PY@it=\textit\def\PY@tc##1{\textcolor[rgb]{0.25,0.50,0.50}{##1}}}
\expandafter\def\csname PY@tok@vc\endcsname{\def\PY@tc##1{\textcolor[rgb]{0.10,0.09,0.49}{##1}}}
\expandafter\def\csname PY@tok@nf\endcsname{\def\PY@tc##1{\textcolor[rgb]{0.00,0.00,1.00}{##1}}}
\expandafter\def\csname PY@tok@kr\endcsname{\let\PY@bf=\textbf\def\PY@tc##1{\textcolor[rgb]{0.00,0.50,0.00}{##1}}}
\expandafter\def\csname PY@tok@nc\endcsname{\let\PY@bf=\textbf\def\PY@tc##1{\textcolor[rgb]{0.00,0.00,1.00}{##1}}}
\expandafter\def\csname PY@tok@vg\endcsname{\def\PY@tc##1{\textcolor[rgb]{0.10,0.09,0.49}{##1}}}
\expandafter\def\csname PY@tok@no\endcsname{\def\PY@tc##1{\textcolor[rgb]{0.53,0.00,0.00}{##1}}}
\expandafter\def\csname PY@tok@s1\endcsname{\def\PY@tc##1{\textcolor[rgb]{0.73,0.13,0.13}{##1}}}
\expandafter\def\csname PY@tok@ow\endcsname{\let\PY@bf=\textbf\def\PY@tc##1{\textcolor[rgb]{0.67,0.13,1.00}{##1}}}
\expandafter\def\csname PY@tok@na\endcsname{\def\PY@tc##1{\textcolor[rgb]{0.49,0.56,0.16}{##1}}}
\expandafter\def\csname PY@tok@sb\endcsname{\def\PY@tc##1{\textcolor[rgb]{0.73,0.13,0.13}{##1}}}
\expandafter\def\csname PY@tok@nt\endcsname{\let\PY@bf=\textbf\def\PY@tc##1{\textcolor[rgb]{0.00,0.50,0.00}{##1}}}
\expandafter\def\csname PY@tok@kn\endcsname{\let\PY@bf=\textbf\def\PY@tc##1{\textcolor[rgb]{0.00,0.50,0.00}{##1}}}
\expandafter\def\csname PY@tok@kc\endcsname{\let\PY@bf=\textbf\def\PY@tc##1{\textcolor[rgb]{0.00,0.50,0.00}{##1}}}
\expandafter\def\csname PY@tok@il\endcsname{\def\PY@tc##1{\textcolor[rgb]{0.40,0.40,0.40}{##1}}}
\expandafter\def\csname PY@tok@sh\endcsname{\def\PY@tc##1{\textcolor[rgb]{0.73,0.13,0.13}{##1}}}
\expandafter\def\csname PY@tok@sc\endcsname{\def\PY@tc##1{\textcolor[rgb]{0.73,0.13,0.13}{##1}}}
\expandafter\def\csname PY@tok@gt\endcsname{\def\PY@tc##1{\textcolor[rgb]{0.00,0.27,0.87}{##1}}}
\expandafter\def\csname PY@tok@o\endcsname{\def\PY@tc##1{\textcolor[rgb]{0.40,0.40,0.40}{##1}}}
\expandafter\def\csname PY@tok@mf\endcsname{\def\PY@tc##1{\textcolor[rgb]{0.40,0.40,0.40}{##1}}}
\expandafter\def\csname PY@tok@w\endcsname{\def\PY@tc##1{\textcolor[rgb]{0.73,0.73,0.73}{##1}}}
\expandafter\def\csname PY@tok@kt\endcsname{\def\PY@tc##1{\textcolor[rgb]{0.69,0.00,0.25}{##1}}}
\expandafter\def\csname PY@tok@mb\endcsname{\def\PY@tc##1{\textcolor[rgb]{0.40,0.40,0.40}{##1}}}
\expandafter\def\csname PY@tok@s\endcsname{\def\PY@tc##1{\textcolor[rgb]{0.73,0.13,0.13}{##1}}}
\expandafter\def\csname PY@tok@go\endcsname{\def\PY@tc##1{\textcolor[rgb]{0.53,0.53,0.53}{##1}}}
\expandafter\def\csname PY@tok@gh\endcsname{\let\PY@bf=\textbf\def\PY@tc##1{\textcolor[rgb]{0.00,0.00,0.50}{##1}}}
\expandafter\def\csname PY@tok@c\endcsname{\let\PY@it=\textit\def\PY@tc##1{\textcolor[rgb]{0.25,0.50,0.50}{##1}}}
\expandafter\def\csname PY@tok@nn\endcsname{\let\PY@bf=\textbf\def\PY@tc##1{\textcolor[rgb]{0.00,0.00,1.00}{##1}}}
\expandafter\def\csname PY@tok@mh\endcsname{\def\PY@tc##1{\textcolor[rgb]{0.40,0.40,0.40}{##1}}}
\expandafter\def\csname PY@tok@k\endcsname{\let\PY@bf=\textbf\def\PY@tc##1{\textcolor[rgb]{0.00,0.50,0.00}{##1}}}
\expandafter\def\csname PY@tok@mi\endcsname{\def\PY@tc##1{\textcolor[rgb]{0.40,0.40,0.40}{##1}}}
\expandafter\def\csname PY@tok@m\endcsname{\def\PY@tc##1{\textcolor[rgb]{0.40,0.40,0.40}{##1}}}
\expandafter\def\csname PY@tok@sd\endcsname{\let\PY@it=\textit\def\PY@tc##1{\textcolor[rgb]{0.73,0.13,0.13}{##1}}}
\expandafter\def\csname PY@tok@kp\endcsname{\def\PY@tc##1{\textcolor[rgb]{0.00,0.50,0.00}{##1}}}
\expandafter\def\csname PY@tok@gu\endcsname{\let\PY@bf=\textbf\def\PY@tc##1{\textcolor[rgb]{0.50,0.00,0.50}{##1}}}
\expandafter\def\csname PY@tok@cp\endcsname{\def\PY@tc##1{\textcolor[rgb]{0.74,0.48,0.00}{##1}}}
\expandafter\def\csname PY@tok@mo\endcsname{\def\PY@tc##1{\textcolor[rgb]{0.40,0.40,0.40}{##1}}}
\expandafter\def\csname PY@tok@gs\endcsname{\let\PY@bf=\textbf}
\expandafter\def\csname PY@tok@nl\endcsname{\def\PY@tc##1{\textcolor[rgb]{0.63,0.63,0.00}{##1}}}
\expandafter\def\csname PY@tok@nb\endcsname{\def\PY@tc##1{\textcolor[rgb]{0.00,0.50,0.00}{##1}}}
\expandafter\def\csname PY@tok@sr\endcsname{\def\PY@tc##1{\textcolor[rgb]{0.73,0.40,0.53}{##1}}}
\expandafter\def\csname PY@tok@cs\endcsname{\let\PY@it=\textit\def\PY@tc##1{\textcolor[rgb]{0.25,0.50,0.50}{##1}}}
\expandafter\def\csname PY@tok@bp\endcsname{\def\PY@tc##1{\textcolor[rgb]{0.00,0.50,0.00}{##1}}}
\expandafter\def\csname PY@tok@gd\endcsname{\def\PY@tc##1{\textcolor[rgb]{0.63,0.00,0.00}{##1}}}
\expandafter\def\csname PY@tok@si\endcsname{\let\PY@bf=\textbf\def\PY@tc##1{\textcolor[rgb]{0.73,0.40,0.53}{##1}}}
\expandafter\def\csname PY@tok@kd\endcsname{\let\PY@bf=\textbf\def\PY@tc##1{\textcolor[rgb]{0.00,0.50,0.00}{##1}}}
\expandafter\def\csname PY@tok@sx\endcsname{\def\PY@tc##1{\textcolor[rgb]{0.00,0.50,0.00}{##1}}}
\expandafter\def\csname PY@tok@gr\endcsname{\def\PY@tc##1{\textcolor[rgb]{1.00,0.00,0.00}{##1}}}
\expandafter\def\csname PY@tok@ge\endcsname{\let\PY@it=\textit}
\expandafter\def\csname PY@tok@gp\endcsname{\let\PY@bf=\textbf\def\PY@tc##1{\textcolor[rgb]{0.00,0.00,0.50}{##1}}}
\expandafter\def\csname PY@tok@se\endcsname{\let\PY@bf=\textbf\def\PY@tc##1{\textcolor[rgb]{0.73,0.40,0.13}{##1}}}
\expandafter\def\csname PY@tok@cm\endcsname{\let\PY@it=\textit\def\PY@tc##1{\textcolor[rgb]{0.25,0.50,0.50}{##1}}}
\expandafter\def\csname PY@tok@err\endcsname{\def\PY@bc##1{\setlength{\fboxsep}{0pt}\fcolorbox[rgb]{1.00,0.00,0.00}{1,1,1}{\strut ##1}}}

\def\PYZbs{\char`\\}
\def\PYZus{\char`\_}
\def\PYZob{\char`\{}
\def\PYZcb{\char`\}}
\def\PYZca{\char`\^}
\def\PYZam{\char`\&}
\def\PYZlt{\char`\<}
\def\PYZgt{\char`\>}
\def\PYZsh{\char`\#}
\def\PYZpc{\char`\%}
\def\PYZdl{\char`\$}
\def\PYZhy{\char`\-}
\def\PYZsq{\char`\'}
\def\PYZdq{\char`\"}
\def\PYZti{\char`\~}
% for compatibility with earlier versions
\def\PYZat{@}
\def\PYZlb{[}
\def\PYZrb{]}
\makeatother


    % Exact colors from NB
    \definecolor{incolor}{rgb}{0.0, 0.0, 0.5}
    \definecolor{outcolor}{rgb}{0.545, 0.0, 0.0}

    % Prevent overflowing lines due to hard-to-break entities
    \sloppy 
    % Slightly bigger margins than the latex defaults
 \hypersetup{pdfauthor={Yigal Weinstein},
            pdftitle={P1: Stroop Effect},
            pdfsubject={data science, Udacity, MOOC},
            pdfkeywords={data science, statistics, probability, machine learning},
            %pdfpagelayout={TwoPageLeft}, % Displays two pages, odd-numbered pages to the left 
            pdfcreator={PdfLaTeX},
            bookmarks={true},            %  A set of Acrobat bookmarks are written
            colorlinks={true},           %  Colors the text of links and anchors. 
            anchorcolor={black},         %  Color for anchor text.
            filecolor={cyan},            %  Color for URLs which open local files.
            menucolor={red},             %  Color for Acrobat menu items.
            runcolor={blue},              %  Color for run links (launch annotations).        
            linkcolor={red!50!black},
            citecolor={blue!50!black},
            %urlcolor={blue!80!black}
            breaklinks=true,  % so long urls are correctly broken across lines
           }           
 
    \geometry{verbose,tmargin=1in,bmargin=1in,lmargin=1in,rmargin=1in}

\newcommand{\sectiontitle}[2]{
\section{#1}
\vspace{-1em}
\mbox{\begin{minipage}{5in}
\textbf{
  \uline{
    #2
  }}
\end{minipage}
}
\vspace{1em}
}

\begin{document}
\maketitle
\section*{Introduction}
So the stage is set with the following description for the Stroop experiment:
\begin{quote}
In a Stroop task, participants are presented with a list of words, with
each word displayed in a color of ink. The participant's task is to say
out loud the color of the ink in which the word is printed. The task has
two conditions: a congruent words condition, and an incongruent words
condition. In the congruent words condition, the words being displayed
are color words whose names match the colors in which they are printed:
for example {\color{red} RED }, {\color{blue} BLUE }. In the incongruent words condition, the words
fisplayed are color words whose names do not match the colors in which
they are printed: for example { \color{yellow}PURPLE }, {\color{green} ORANGE}. In each case, we measure
the time it takes to name the ink colors in equally-sized lists. Each
participant will go through and record a time from each condition.
\cite{Udacity-P1-Stroop}
\end{quote}
This project attempts to follow the outline provided in \cite{Udacity-P1-Stroop}
and the project rubrick set forth in \cite{Udacity-P1-rubrick}.
\sectiontitle{Identify variables in the experiment}
{
\label{sec:question1}
What is our independent variable? What is our dependent variable?}
  incongruency between the color of a word, and the symantic meaning of the
  word.  As example the word {\color{red}`RED'} is a congruent example where the
  word `red' has the color of its semantic meaning, i.e. the color it
  represents, and {\color{green}`BLUE'} is an incongruent case where the word
  `blue' which has the symantic meaning of the color blue is instead the color
  green. Frome here on out the terms congruent color selection, CCS, and
incongruent color selection, ICS, will be used to denote the two conditions the
dependent variable may take that of a word representing a color being colored
with or with a different color respectively.

The dependent variable is the time, in seconds, it takes to read through an
equal sized lists of CCS or ICS words.  
\sectiontitle{Establish a hypothesis and statistical test}{\label{sec:question2}
What is an appropriate set of hypotheses for this task? What kind of statistical
test do you expect to perform? 
Justify your choices.}

The null hypothesis, \(H_0\) is the statement that the ICS task will take less
or an equal amount of time to the CCS task. The alternative hypothesis, \(H_A\)
is the statement that the mean population time to read an ICS list will be
greater than an equally sized CCS list. The null and alternative hypothesis
choice stems from a need to test quantatively the validity and extent of an
already well respected phenomenon the Stroop effect. The Stroop effect being
defined to mean that cognitive incongruences will lead to an increase in the
reaction time of a task.\\ It is true that this single tailed objective may not
provide insight if indeed the reaction time for ICS word lists take, on the
average, less time to complete, however this is a test regarding the scope and
veracity of the Stroop effect which has been shown in many previous quantative
tests to have general validity. If indeed the ICS task takes less time it is
enough that \(H_0\) should be kept to cause doubt regarding the validity of
either the test or the Stroop effect.  That is due to the large body of work
already done to test the effect if \(H_0\) is kept there is either something
wrong with the experiment or the Stroop effect has either more limitations on
generalizability than expected or there are indeed real issues with it's
veracity. So to sum things up via a set of equations:

\begin{align}
H_0:\; \mu_{ICS} \le \mu_{CCS} \\
H_A:\; \mu_{ICS} > \mu_{CCS}
\end{align}
As the two sets of data are correlated, the same subjects are used to take both
the ICS and CCS tests, the proper test to use is a dependent t-test for paired
samples.  Given the hypothesis devised above the single tail test will be used.

\sectiontitle{Report descriptive statistics}{
  \label{sec:question3}
Report some descriptive statistics regarding this dataset.
Include at least one measure of central tendency and at least one measure of variability.}

In Figure \ref{fig:descriptive_data_describe} is printed the mean, sample standard deviation, minimum values,
25th, 50th - median, 75th percentiles, and maximums for both the CCS
and ICS samples, all values are in seconds. 
\begin{figure}[ht]
  \centering
  \input{data_describe}
  \caption{The source code for the table may be found:
    \href{https://github.com/8leggedunicorn/p1_stroop_effect/blob/master/p1_stroop_effect.py\#L1-L13}
        {p1\_stroop\_effect.py}.
      The above table was generated with help from official PANDAS API
      documentation as follows:
    \cite{PANDAS-csv}, \cite{PANDAS-to-latex}, \cite{PANDAS-describe-df}}
  \label{fig:descriptive_data_describe}
\end{figure}
\begin{center}
\end{center}
To measure the central tendency consider the mean, and the median - which in this
case is simply the 50th percentile. Notice the fact that for both CCS and ICS
the median and mean are nearly identical, for the CCS case it's less than one
third of a second difference, and for the ICS case the difference between these
two measure of central tendency is only slightly greater than a second. This
points to both distribution being evenly distributed around the mean. As we can
see for nearly all percentiles, minimum and maximum, and mean values the CCS
times are roughly two thirds of the value of the ICS times, and all times for
every category are significantly longer for the ICS sample relative to the CCS.

An indication of the relative spread can be observed by looking at the sample
standard deviations. the ICS case is slightly more spread out than the CCS case
\(4.797s\) to \(3.559s\) respectively, i.e slightly over a second more for the
ICS sample. To gain a better descriptive understanding of the spread we can also
look at the range, i.e.  \(R = x_{max} - x_{min}\) of the two samples:

\begin{align*}
R_{CCS} = x_{CCS,max} - x_{CCS,min} = 22.328000 - 8.630000 = 13.698\\
R_{ICS} = x_{ICS,max} - x_{ICS,min} = 35.255000 - 15.687000 = 19.568
\end{align*}

From both the sample standard deviation and the range we can see that the spread
of the ICS sample is greater both in terms of the range, i.e.~outliers, and of
the sample standard deviation:

\sectiontitle{Plot the data}
{\label{sec:question4}
Provide one or two visualizations that show the distribution of the sample data.
Write one or two sentences noting what you observe about the plot or plots.}

Figure \ref{fig:bar-graph-ccs-vs-ics-times} points to the same ideas described
in the descriptive statistics presented in the previous section. There are no
real outliers for the CCS case, which shows up in the difference in the standard
deviations of the two samples above, and that the mean and median for the CCS
case are very close together.  What is interesting to note, however is that the
two greatest times for the ICS are from subjects who take no where near the
longest for the CCS trial. Subject 20 having the second greatest ICS time isn't
even to the right of the mean.  This graph also explicitly shows that each
subject takes a greater ICS time than their CCS time, there being no exceptions.
\begin{figure}[ht]
  \centering
  \includegraphics{bar_graph.pdf}
  \caption{Bar graph of the time to complete the CCS and ICS list for each
    subject.  The subject is labled in order of their position in the original
    data, however the data is ordered by CCS time, from least to greatest.  The
    code for the plot can be found in
    \href{https://github.com/8leggedunicorn/p1_stroop_effect/blob/master/p1_stroop_effect.py}{p1\_stroop\_effect.py}
    and demos and documentation from the official Matplotlib website were used
    to generate it, 
    \cite{Matplotlib-bar}, 
    \cite{Matplotlib-colors},
    \cite{Matplotlib-legend-transparent}, 
    \cite{Matplotlib-pdf},
    \cite{Matplotlib-pyplot}, 
    \cite{Python-comprehension}}
  \label{fig:bar-graph-ccs-vs-ics-times}
\end{figure}
One might think naively especially in the case of the CCS descriptive statistics
that both data sets show an evenly distributed nature, however looking at the
two histogram graphs, one of the CCS and ICS results we see this is hardly the
picture.  Please look to figures \ref{fig:ICS-histogram} to view such plots:
\begin{figure}[ht]
  \centering
  \includegraphics[scale=0.75]{histogram-ics}
  \caption{The graph above is a histogram of the ICS data.  Note how it is
  skewed heavily to the left.  This is due to the two outliers with times
  greater than $32s$.}
  \label{fig:ICS-histogram}
\end{figure}
At first glimpse of either of these two histograms it becomes clear that they
are both heavily skewed to the left, especially in the case of the ICS data set.
\sectiontitle{Perform the statistical test and interpret your results}{
\label{sec:question5}
Now, perform the statistical test and report your results.
What is your confidence level and your critical statistic value?
Do you reject the null hypothesis or fail to reject it?
Come to a conclusion in terms of the experiment task. Did the results match up with your expectations?} 

\sectiontitle{Digging deeper and extending the investigation}{
\label{sec:question6}
  What do you think is responsible for the effects observed?
  Can you think of an alternative or similar task that would result in a similar
  effect?  Some research about the problem will be helpful for thinking about these two questions!}
\printbibliography
\end{document}
